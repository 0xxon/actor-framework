\section{Platform-Independent Type System}
\label{Sec::TypeSystem}

\lib provides a fully network transparent communication between actors.
Thus, \lib needs to serialize and deserialize messages.
Unfortunately, this is not possible using the RTTI system of C++.
\lib uses its own RTTI based on the class \lstinline^uniform_type_info^, since it is not possible to extend \lstinline^std::type_info^.

Unlike \lstinline^std::type_info::name()^, \lstinline^uniform_type_info::name()^ is guaranteed to return the same name on all supported platforms. Furthermore, it allows to create an instance of a type by name.

\begin{lstlisting}
// creates a signed, 32 bit integer
cppa::object i = cppa::uniform_typeid<int>()->create();
\end{lstlisting}

However, you should rarely if ever need to use \lstinline^object^ or \lstinline^uniform_type_info^.

\subsection{User-Defined Data Types in Messages}
\label{Sec::TypeSystem::UserDefined}

All user-defined types must be explicitly ``announced'' so that \lib can (de)serialize them correctly, as shown in the example below.

\begin{lstlisting}
#include "cppa/cppa.hpp"
using namespace cppa;

struct foo { int a; int b; };

int main() {
  announce<foo>(&foo::a, &foo::b);
  // ... foo can now safely be used in messages ...
}
\end{lstlisting}

Without the \lstinline^announce^ function call, the example program would terminate with an exception, because \lib rejects all types without available runtime type information.

\lstinline^announce()^ takes the class as template parameter and pointers to all members (or getter/setter pairs) as arguments.
This works for all primitive data types and STL compliant containers.
See the announce examples 1 -- 4 of the standard distribution for more details.

Obviously, there are limitations.
You have to implement serialize/deserialize by yourself if your class does implement an unsupported data structure.
See \lstinline^announce_example_5.cpp^ in the examples folder.
