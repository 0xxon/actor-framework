\section{Actors}
\label{Sec::Actors}

\lib provides several actor implementations, each covering a particular use case.
The class \lstinline^local_actor^ is the base class for all implementations, except for (remote) proxy actors.
Hence, \lstinline^local_actor^ provides a common interface for actor operations like trapping exit messages or finishing execution.
The default actor implementation in \lib is event-based.
Event-based actors have a very small memory footprint and are thus very lightweight and scalable.
Context-switching actors are used for actors that make use of the blocking API (see Section \ref{Sec::BlockingAPI}), but do not need to run in a separate thread.
Context-switching and event-based actors are scheduled cooperatively in a thread pool.
Thread-mapped actors can be used to opt-out of this cooperative scheduling.

\subsection{Implicit \texttt{self} Pointer}

When using a function or functor to implement an actor, the first argument \emph{can} be used to capture a pointer to the actor itself.
The type of this pointer is \lstinline^event_based_actor*^ per default and \lstinline^blocking_actor*^ when using the \lstinline^blocking_api^ flag.
When dealing with typed actors, the types are \lstinline^typed_event_based_actor<...>*^ and \lstinline^typed_blocking_actor<...>*^.

\clearpage
\subsection{Interface}
\label{Sec::Actors::Interfaces}

\begin{lstlisting}
class local_actor;
\end{lstlisting}

{\small
\begin{tabular*}{\textwidth}{m{0.45\textwidth}m{0.5\textwidth}}
  \multicolumn{2}{m{\linewidth}}{\large{\textbf{Member functions}}\vspace{3pt}} \\
  \\
  \multicolumn{2}{l}{\textbf{Observers}\vspace{3pt}} \\
  \hline
  \lstinline^actor_addr address()^ & Returns the address of this actor \\
  \hline
  \lstinline^bool trap_exit()^ & Checks whether this actor traps exit messages \\
  \hline
  \lstinline^message& current_message()^ & Returns the currently processed message\newline\textbf{Warning}: Only set during callback invocation; calling this function after forwarding the message or while not in a callback is undefined behavior \\
  \hline
  \lstinline^actor_addr& current_sender()^ & Returns the sender of the current message\newline\textbf{Warning}: Only set during callback invocation; calling this function after forwarding the message or while not in a callback is undefined behavior \\
  \hline
  \lstinline^vector<group> joined_groups()^ & Returns all subscribed groups \\
  \hline
  \\
  \multicolumn{2}{l}{\textbf{Modifiers}\vspace{3pt}} \\
  \hline
  \lstinline^quit(uint32_t reason = normal)^ & Finishes execution of this actor \\
  \hline
  \lstinline^void trap_exit(bool enabled)^ & Enables or disables trapping of exit messages \\
  \hline
  \lstinline^void join(const group& g)^ & Subscribes to group \lstinline^g^ \\
  \hline
  \lstinline^void leave(const group& g)^ & Unsubscribes group \lstinline^g^ \\
  \hline
  \lstinline^void on_sync_failure(auto fun)^ & Sets a handler, i.e., a functor taking no arguments, for unexpected synchronous response messages (default action is to kill the actor for reason \lstinline^unhandled_sync_failure^) \\
  \hline
  \lstinline^void monitor(actor whom)^ & Unidirectionally monitors \lstinline^whom^ (see Section \ref{Sec::Management::Monitors}) \\
  \hline
  \lstinline^void demonitor(actor whom)^ & Removes a monitor from \lstinline^whom^ \\
  \hline
  \lstinline^bool has_sync_failure_handler()^ & Checks wheter this actor has a user-defined sync failure handler \\
  \hline
  \lstinline^template <class F>^ \lstinline^void set_exception_handler(F f)^ & Sets a custom handler for uncaught exceptions \\
  \hline
\end{tabular*}
}
