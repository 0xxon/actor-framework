\section{Synchronous Communication}
\label{Sec::Sync}

\lib supports both asynchronous and synchronous communication.
The member functions \lstinline^sync_send^ and \lstinline^sync_send_tuple^ send synchronous request messages.

\begin{lstlisting}
template<typename... Args>
__unspecified__ sync_send(actor whom, Args&&... what);

__unspecified__ sync_send_tuple(actor whom, message what);

template<typename Duration, typename... Args>
__unspecified__ timed_sync_send(actor whom,
                                Duration timeout,
                                Args&&... what);

template<typename Duration, typename... Args>
__unspecified__ timed_sync_send_tuple(actor whom,
                                      Duration timeout,
                                      message what);
\end{lstlisting}

A synchronous message is sent to the receiving actor's mailbox like any other asynchronous message.
The response message, on the other hand, is treated separately.

The difference between \lstinline^sync_send^ and \lstinline^timed_sync_send^ is how timeouts are handled.
The behavior of \lstinline^sync_send^ is analogous to \lstinline^send^, i.e., timeouts are specified by using \lstinline^after(...)^ statements (see \ref{Sec::Receive::Timeouts}).
When using \lstinline^timed_sync_send^ function, \lstinline^after(...)^ statements are ignored and the actor will receive a \lstinline^sync_timeout_msg^ after the given duration instead.

\subsection{Error Messages}

When using synchronous messaging, \lib's runtime environment will send ...

\begin{itemize}
\item if the receiver is not alive:\newline\lstinline^sync_exited_msg { actor_addr source; std::uint32_t reason; };^
%\item \lstinline^{'VOID'}^ if the receiver handled the message but did not respond to it
\item if a message send by \lstinline^timed_sync_send^ timed out: \lstinline^sync_timeout_msg^
\end{itemize}

\clearpage
\subsection{Receive Response Messages}

When sending a synchronous message, the response handler can be passed by either using \lstinline^then^ (event-based actors) or \lstinline^await^ (blocking actors).

\begin{lstlisting}
void foo(event_based_actor* self, actor testee) {
  // testee replies with a string to 'get'
  self->sync_send(testee, atom("get")).then(
    [=](const std::string& str) {
      // handle str
    },
    after(std::chrono::seconds(30)) >> [=]() {
      // handle error
    }
  );
);
\end{lstlisting}

Similar to \lstinline^become^, the \lstinline^then^ function modifies an actor's behavior stack.
However, it is used as ``one-shot handler'' and automatically returns to the previous behavior afterwards.

\subsection{Synchronous Failures and Error Handlers}

An unexpected response message, i.e., a message that is not handled by given behavior, will invoke the actor's \lstinline^on_sync_failure^ handler.
The default handler kills the actor by calling \lstinline^self->quit(exit_reason::unhandled_sync_failure)^.
The handler can be overridden by calling \lstinline^self->on_sync_failure(/*...*/)^.

Unhandled timeout messages trigger the \lstinline^on_sync_timeout^ handler.
The default handler kills the actor for reason \lstinline^exit_reason::unhandled_sync_failure^.
It is possible set both error handlers by calling \lstinline^self->on_sync_timeout_or_failure(/*...*)^.

\begin{lstlisting}
void foo(event_based_actor* self, actor testee) {
  // testee replies with a string to 'get'
  // set handler for unexpected messages
  self->on_sync_failure = [] {
    aout << "received: " << to_string(self->last_dequeued()) << endl;
  };
  // set handler for timeouts
  self->on_sync_timeout = [] {
    aout << "timeout occured" << endl;
  };
  // set response handler by using "then"
  timed_sync_send(testee, std::chrono::seconds(30), atom("get")).then(
    [=](const std::string& str) { /* handle str */ }
  );
\end{lstlisting}
