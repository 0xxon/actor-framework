\section{Common Pitfalls}
\label{Sec::Pitfalls}

\subsection{Event-Based API}

\begin{itemize}
\item The functions \lstinline^become^ and \lstinline^handle_response^ do not block, i.e., always return immediately.
Thus, you should \textit{always} capture by value in event-based actors, because all references on the stack will cause undefined behavior if a lambda is executed.
\end{itemize}

\subsection{Mixing Event-Based and Blocking API}

\begin{itemize}
\item Blocking \libcppa function such as \lstinline^receive^ will \emph{throw an exception} if accessed from an event-based actor.
%To catch as many errors as possible at compile-time, \libcppa will produce an error if \lstinline^receive^ is called and the \lstinline^this^ pointer is set and points to an event-based actor.

\item Context-switching and thread-mapped actors \emph{can} use the \lstinline^become^ API.
Whenever a non-event-based actor calls \lstinline^become()^ for the first time, it will create a behavior stack and execute it until the behavior stack is empty.
Thus, the \textit{initial} \lstinline^become^ blocks until the behavior stack is empty, whereas all subsequent calls to \lstinline^become^ will return immediately.
Related functions, e.g., \lstinline^sync_send(...).then(...)^, behave the same, as they manipulate the behavior stack as well.
\end{itemize}

\subsection{Synchronous Messages}

\begin{itemize}

\item
\lstinline^send(self->last_sender(), ...)^ is \textbf{not} equal to \lstinline^reply(...)^.
The two functions \lstinline^receive_response^ and \lstinline^handle_response^ will only recognize messages send via either \lstinline^reply^ or \lstinline^reply_tuple^.

\item
A handle returned by \lstinline^sync_send^ represents \emph{exactly one} response message.
Therefore, it is not possible to receive more than one response message.
Calling \lstinline^reply^ more than once will result in lost messages and calling \lstinline^handle_response^ or \lstinline^receive_response^ more than once on a future will throw an exception.

\item
The future returned by \lstinline^sync_send^ is bound to the calling actor.
It is not possible to transfer such a future to another actor.
Calling \lstinline^receive_response^ or \lstinline^handle_response^ for a future bound to another actor is undefined behavior.

\end{itemize}

\subsection{Sending Messages}

\begin{itemize}

\item
\lstinline^send(whom, ...)^ is syntactic sugar for \lstinline^whom << make_any_tuple(...)^.
Hence, a message sent via \lstinline^send(whom, self->last_dequeued())^ will not yield the expected result, since it wraps \lstinline^self->last_dequeued()^ into another \lstinline^any_tuple^ instance.
The correct way of forwarding messages is \lstinline^self->forward_to(whom)^.

\end{itemize}

\clearpage

\subsection{Sharing}

\begin{itemize}
\item It is strongly recommended to \textbf{not} share states between actors.
In particular, no actor shall ever access member variables or member functions of another actor.
Accessing shared memory segments concurrently can cause undefined behavior that is incredibly hard to find and debug.
However, sharing \textit{data} between actors is fine, as long as the data is \textit{immutable} and all actors access the data only via smart pointers such as \lstinline^std::shared_ptr^.
Nevertheless, the recommended way of sharing informations is message passing.
Sending data to multiple actors does \textit{not} result in copying the data several times.
Read Section \ref{Sec::Tuples} to learn more about \libcppa's copy-on-write optimization for tuples.
\end{itemize}

\subsection{Constructors of Class-based Actors}

\begin{itemize}
\item During constructor invocation, \lstinline^self^ does \textbf{not} point to \lstinline^this^. It points to the invoking actor instead.
\item You should \textbf{not} send or receive messages in a constructor or destructor.
\end{itemize}
