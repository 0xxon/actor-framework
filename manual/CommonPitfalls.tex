\section{Common Pitfalls}
\label{Sec::Pitfalls}

\subsection{Event-Based API}

\begin{itemize}
\item The functions \lstinline^become^ and \lstinline^handle_response^ do not block, i.e., always return immediately.
Thus, one should \textit{always} capture by value in lambda expressions, because all references on the stack will cause undefined behavior if the lambda expression is executed.
\end{itemize}

\subsection{Synchronous Messages}

\begin{itemize}

\item
A handle returned by \lstinline^sync_send^ represents \emph{exactly one} response message.
Therefore, it is not possible to receive more than one response message.
%Calling \lstinline^reply^ more than once will result in lost messages and calling \lstinline^handle_response^ or \lstinline^receive_response^ more than once on a future will throw an exception.

\item
The handle returned by \lstinline^sync_send^ is bound to the calling actor.
It is not possible to transfer a handle to a response to another actor.

\end{itemize}

\subsection{Sending Messages}

\begin{itemize}

\item
\lstinline^send(whom, ...)^ is defined as \lstinline^send_tuple(whom, make_message(...))^.
Hence, a message sent via \lstinline^send(whom, self->last_dequeued())^ will not yield the expected result, since it wraps \lstinline^self->last_dequeued()^ into another \lstinline^message^ instance.
The correct way of forwarding messages is \lstinline^self->forward_to(whom)^.

\end{itemize}

\subsection{Sharing}

\begin{itemize}
\item It is strongly recommended to \textbf{not} share states between actors.
In particular, no actor shall ever access member variables or member functions of another actor.
Accessing shared memory segments concurrently can cause undefined behavior that is incredibly hard to find and debug.
However, sharing \textit{data} between actors is fine, as long as the data is \textit{immutable} and its lifetime is guaranteed to outlive all actors.
The simplest way to meet the lifetime guarantee is by storing the data in smart pointers such as \lstinline^std::shared_ptr^.
Nevertheless, the recommended way of sharing informations is message passing.
Sending the same message to multiple actors does not result in copying the data several times.
\end{itemize}

\subsection{Constructors of Class-based Actors}

\begin{itemize}
\item You should \textbf{not} try to send or receive messages in a constructor or destructor, because the actor is not fully initialized at this point.
\end{itemize}
