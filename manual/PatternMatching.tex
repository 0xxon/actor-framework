\section{Pattern Matching}
\label{Sec::PatternMatching}

C++ does not provide pattern matching facilities.
A general pattern matching solution for arbitrary data structures would require a language extension.
Hence, we decided to restrict our implementation to tuples, to be able to use an internal domain-specific language approach.

\subsection{Basics}
\label{Sec::PatternMatching::Basics}

A match expression begins with a call to the function \lstinline^on^, which returns an intermediate object providing the member function \lstinline^when^ and \lstinline^operator>>^.
The right-hand side of the operator denotes a callback, usually a lambda expression, that should be invoked if a tuple matches the types given to \lstinline^on^,
as shown in the example below.

\begin{lstlisting}
on<int>() >> [](int i) { /*...*/ }
on<int, float>() >> [](int i, float f) { /*...*/ }
on<int, int, int>() >> [](int a, int b, int c) { /*...*/ }
\end{lstlisting}

The result of \lstinline^operator>>^ is a \textit{match statement}.
A partial function can consist of any number of match statements.
At most one callback is invoked, since the evaluation stops at the first match.

\begin{lstlisting}
partial_function fun {
  on<int>() >> [](int i) {
    // case1
  },
  on<int>() >> [](int i) {
    // case2; never invoked, since case1 always matches first
  }
};
\end{lstlisting}

The function ``\lstinline^on^'' can be used in two ways.
Either with template parameters only or with function parameters only.
The latter version deduces all types from its arguments and matches for both type and value.
To match for any value of a given type, ``\lstinline^val^'' can be used, as shown in the following example.

\begin{lstlisting}
on(42) >> [](int i) { assert(i == 42); }
on("hello world") >> [] { /* ... */ }
on("print", val<std::string>) >> [](const std::string& what) {
  // ...
}
\end{lstlisting}

\textbf{Note:} The given callback can have less arguments than the pattern.
But it is only allowed to skip arguments from left to right.

\begin{lstlisting}
on<int, float, double>() >> [](double) { /*...*/ }             // ok
on<int, float, double>() >> [](float, double) { /*...*/ }      // ok
on<int, float, double>() >> [](int, float, double) { /*...*/ } // ok

on<int, float, double>() >> [](int i) { /*...*/ } // compiler error
\end{lstlisting}

\subsection{Reducing Redundancy with ``\lstinline^arg_match^'' and ``\lstinline^on_arg_match^''}

Our previous examples always used the most verbose form, which is quite redundant, since you have to type the types twice -- as template parameter and as argument type for the lambda.
To avoid such redundancy, \lstinline^arg_match^ can be used as last argument to the function \lstinline^on^.
This causes the compiler to deduce all further types from the signature of the given callback.

\begin{lstlisting}
on<int, int>() >> [](int a, int b) { /*...*/ }
// is equal to:
on(arg_match) >> [](int a, int b) { /*...*/ }
\end{lstlisting}

Note that the second version does call \lstinline^on^ without template parameters.
Furthermore, \lstinline^arg_match^ must be passed as last parameter.
If all types should be deduced from the callback signature, \lstinline^on_arg_match^ can be used.
It is equal to \lstinline^on(arg_match)^.

\begin{lstlisting}
on_arg_match >> [](const std::string& str) { /*...*/ }
\end{lstlisting}

\subsection{Atoms}
\label{Sec::PatternMatching::Atoms}

Assume an actor provides a mathematical service for integers.
It takes two arguments, performs a predefined operation and returns the result.
It cannot determine an operation, such as multiply or add, by receiving two operands.
Thus, the operation must be encoded into the message.
The Erlang programming language introduced an approach to use non-numerical
constants, so-called \textit{atoms}, which have an unambiguous, special-purpose type and do not have the runtime overhead of string constants.
Atoms are mapped to integer values at compile time in \libcppa.
This mapping is guaranteed to be collision-free and invertible, but limits atom literals to ten characters and prohibits special characters.
Legal characters are ``\lstinline[language=C++]^_0-9A-Za-z^'' and the whitespace character.
Atoms are created using the \lstinline^constexpr^ function \lstinline^atom^, as the following example illustrates.

\begin{lstlisting}
on(atom("add"), arg_match) >> [](int a, int b) { /*...*/ },
on(atom("multiply"), arg_match) >> [](int a, int b) { /*...*/ },
// ...
\end{lstlisting}

\textbf{Note}: The compiler cannot enforce the restrictions at compile time, except for a length check.
The assertion \lstinline^atom("!?") != atom("?!")^ is not true, because each invalid character is mapped to the whitespace character.

\clearpage
\subsection{Wildcards}
\label{Sec::PatternMatching::Wildcards}

The type \lstinline^anything^ can be used as wildcard to match any number of any types.
%The \lstinline^constexpr^ value \lstinline^any_vals^ can be used as function argument if \lstinline^on^ is used without template paremeters.
A pattern created by \lstinline^on<anything>()^ or its alias \lstinline^others()^ is useful to define a default case.
For patterns defined without template parameters, the \lstinline^constexpr^ value \lstinline^any_vals^ can be used as function argument.
The constant \lstinline^any_vals^ is of type \lstinline^anything^ and is nothing but syntactic sugar for defining patterns.

\begin{lstlisting}[language=C++]
on<int, anything>() >> [](int i) {
  // tuple with int as first element
},
on(any_vals, arg_match) >> [](int i) {
  // tuple with int as last element
  // "on(any_vals, arg_match)" is equal to "on(anything{}, arg_match)"
},
others() >> [] {
  // everything else (default handler)
  // "others()" is equal to "on<anything>()" and "on(any_vals)"
}
\end{lstlisting}

\subsection{Guards}

Guards can be used to constrain a given match statement by using placeholders, as the following example illustrates.

\begin{lstlisting}
using namespace cppa::placeholders; // contains _x1 - _x9

on<int>().when(_x1 % 2 == 0) >> [] {
  // int is even
},
on<int>() >> [] {
  // int is odd
}
\end{lstlisting}

Guard expressions are a lazy evaluation technique.
The placeholder \lstinline^_x1^ is substituted with the first value of a given tuple.
All binary comparison and arithmetic operators are supported, as well as \lstinline^&&^ and \lstinline^||^.
In addition, there are three functions designed to be used in guard expressions: \lstinline^gref^ (``guard reference''), \lstinline^gval^ (``guard value''), and \lstinline^gcall^ (``guard function call'').
The function \lstinline^gref^ creates a reference wrapper, while \lstinline^gval^ encloses a value.
It is similar to \lstinline^std::ref^ but it is always \lstinline^const^ and ``lazy''.
A few examples to illustrate some pitfalls:

\begin{lstlisting}
int val = 42;

on<int>().when(_x1 == val)           // (1) matches if _x1 == 42
on<int>().when(_x1 == gref(val))     // (2) matches if _x1 == val
on<int>().when(_x1 == std::ref(val)) // (3) ok, because of placeholder
others().when(gref(val) == 42)       // (4) matches everything
                                     //     as long as val == 42
others().when(std::ref(val) == 42)   // (5) compiler error
\end{lstlisting}

Statement \texttt{(5)} is evaluated immediately and returns a boolean, whereas statement \texttt{(4)} creates a valid guard expression.
Thus, you should always use \lstinline^gref^ instead of \lstinline^std::ref^ to avoid errors.

The second function, \lstinline^gcall^, encapsulates a function call.
Its usage is similar to \lstinline^std::bind^, but there is also a short version for unary functions: \lstinline^gcall(fun, _x1)^ is equal to \lstinline^_x1(fun)^.

\begin{lstlisting}
auto vec_sorted = [](const std::vector<int>& vec) {
  return std::is_sorted(vec.begin(), vec.end());
};

on<std::vector<int>>().when(gcall(vec_sorted, _x1)) // is equal to:
on<std::vector<int>>().when(_x1(vec_sorted)))
\end{lstlisting}

\subsubsection{Placeholder Interface}

\begin{lstlisting}
template<int X>
struct guard_placeholder;
\end{lstlisting}

\begin{tabular*}{\textwidth}{m{0.3\linewidth}m{0.7\linewidth}}
  \multicolumn{2}{m{\linewidth}}{\large{\textbf{Member functions} \small{(\lstinline^x^ represents the value at runtime, \lstinline^y^ represents an iterable container)}}\vspace{3pt}} \\
  \hline
  \lstinline^size()^ & Returns \lstinline^x.size()^ \\
  \hline
  \lstinline^empty()^ & Returns \lstinline^x.empty()^ \\
  \hline
  \lstinline^not_empty()^ & Returns \lstinline^!x.empty()^ \\
  \hline
  \lstinline^front()^ & Returns an \lstinline^option^ (see Section \ref{Appendix::Option}) to \lstinline^x.front()^  \\
  \hline
  \lstinline^in(y)^ & Returns \lstinline^true^ if \lstinline^y^ contains \lstinline^x^, \lstinline^false^ otherwise\\
  \hline
  \lstinline^not_in(y)^ & Returns \lstinline^!in(y)^  \\
  \hline
  \\
\end{tabular*}

\subsubsection{Examples for Guard Expressions}

\begin{lstlisting}
using namespace std;
typedef vector<int> ivec;

vector<string> strings{"abc", "def"};

on_arg_match.when(_x1.front() == 0) >> [](const ivec& v) {
  // note: we don't have to check whether _x1 is empty in our guard,
  //       because '_x1.front()' returns an option for a 
  //       reference to the first element
  assert(v.size() >= 1);
  assert(v.front() == 0);
},
on<int>().when(_x1.in({10, 20, 30})) >> [](int i) {
  assert(i == 10 || i == 20 || i == 30);
},
on<string>().when(_x1.not_in(strings)) >> [](const string& str) {
  assert(str != "abc" && str != "def");
},
on<string>().when(_x1.size() == 10) >> [](const string& str) {
  // ...
}
\end{lstlisting}

\subsection{Projections and Extractors}

Projections perform type conversions or extract data from a given input.
If a callback expects an integer but the received message contains a string, a projection can be used to perform a type conversion on-the-fly.
This conversion should be free of side-effects and, in particular, shall not throw exceptions, because a failed projection is not an error.
A pattern simply does not match if a projection failed.
Let us have a look at a simple example.

\begin{lstlisting}
auto intproj = [](const string& str) -> option<int> {
  char* endptr = nullptr;
  int result = static_cast<int>(strtol(str.c_str(), &endptr, 10));
  if (endptr != nullptr && *endptr == '\0') return result;
  return {};
};
partial_function fun {
  on(intproj) >> [](int i) {
    // case 1: successfully converted a string
  },
  on_arg_match >> [](const string& str) {
    // case 2: str is not an integer
  }
};
\end{lstlisting}

The lambda \lstinline^intproj^ is a \lstinline^string^ $\Rightarrow$ \lstinline^int^ projection, but note that it does not return an integer.
It returns \lstinline^option<int>^, because the projection is not guaranteed to always succeed.
An empty \lstinline^option^ indicates, that a value does not have a valid mapping to an integer.
A pattern does not match if a projection failed.

\textbf{Note}: Functors used as projection must take exactly one argument and must return a value.
The types for the pattern are deduced from the functor's signature.
If the functor returns an \lstinline^option<T>^, then \lstinline^T^ is deduced.

%\clearpage
%\subsection{Match Expressions}










