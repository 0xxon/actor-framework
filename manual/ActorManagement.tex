\section{Management \& Error Detection}

\libcppa adapts Erlang's well-established fault propagation model.
It allows to build actor subsystem in which either all actors are alive or have collectively failed.

\subsection{Links}

Linked actors monitor each other.
An actor sends an exit message to all of its links as part of its termination.
The default behavior for actors receiving such an exit message is to die for the same reason, if the exit reason is non-normal.
Actors can \textit{trap} exit messages to handle them manually.

\begin{lstlisting}
actor worker = ...;
// receive exit messages as regular messages
self->trap_exit(true);
// monitor spawned actor
self->link_to(worker);
// wait until worker exited
self->become (
  [](const exit_msg& e) >> [=] {
    if (e.reason == exit_reason::normal) {
      // worker finished computation
    else {
      // worker died unexpectedly
    }
  }
);
\end{lstlisting}

\subsection{Monitors}
\label{Sec::Management::Monitors}

A monitor observes the lifetime of an actor.
Monitored actors send a down message to all observers as part of their termination.
Unlike exit messages, down messages are always treated like any other ordinary message.
An actor will receive one down message for each time it called \lstinline^self->monitor(...)^, even if it adds a monitor to the same actor multiple times.

\begin{lstlisting}
actor worker = ...;
// monitor spawned actor
self->monitor(worker);
// wait until worker exited
self->become (
  on(const down_msg& d) >> [] {
    if (d.reason == exit_reason::normal) {
      // worker finished computation
    } else {
      // worker died unexpectedly
    }
  }
);
\end{lstlisting}

\subsection{Error Codes}

All error codes are defined in the namespace \lstinline^cppa::exit_reason^.
To obtain a string representation of an error code, use \lstinline^cppa::exit_reason::as_string(uint32_t)^.

\begin{tabular*}{\textwidth}{m{0.35\textwidth}m{0.08\textwidth}m{0.5\textwidth}}
  \hline
  \lstinline^normal^ & 1 & Actor finished execution without error \\
  \hline
  \lstinline^unhandled_exception^ & 2 & Actor was killed due to an unhandled exception \\
  \hline
  \lstinline^unhandled_sync_failure^ & 4 & Actor was killed due to an unexpected synchronous response message \\
  \hline
  \lstinline^unhandled_sync_timeout^ & 5 & Actor was killed, because no timeout handler was set and a synchronous message timed out \\
  \hline
  \lstinline^user_shutdown^ & 16 & Actor was killed by a user-generated event \\
  \hline
  \lstinline^remote_link_unreachable^ & 257 & Indicates that a remote actor became unreachable, e.g., due to connection error \\
  \hline
  \lstinline^user_defined^ & 65536 & Minimum value for user-defined exit codes \\
  \hline
\end{tabular*}

\subsection{Attach Cleanup Code to an Actor}

Actors can attach cleanup code to other actors.
This code is executed immediately if the actor has already exited.
Keep in mind that \lstinline^self^ refers to the currently running actor.
Thus, \lstinline^self^ refers to the terminating actor and not to the actor that attached a functor to it.

\begin{lstlisting}
auto worker = spawn(...);
actor observer = self;
// "monitor" spawned actor
worker->attach_functor([observer](std::uint32_t reason) {
  // this callback is invoked from worker
  anon_send(observer, atom("DONE"));
});
// wait until worker exited
self->become (
  on(atom("DONE")) >> [] {
    // worker terminated
  }
);
\end{lstlisting}

\textbf{Note}: It is possible to attach code to remote actors, but the cleanup code will run on the local machine.
